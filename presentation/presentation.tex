%%%%%%%%%%%%%%%%%%%%%%%%%%%%%%%%%%%%%%%%%
% Beamer Presentation
% LaTeX Template
% Version 1.0 (10/11/12)
%
% This template has been downloaded from:
% http://www.LaTeXTemplates.com
%
% License:
% CC BY-NC-SA 3.0 (http://creativecommons.org/licenses/by-nc-sa/3.0/)
%
%%%%%%%%%%%%%%%%%%%%%%%%%%%%%%%%%%%%%%%%%

%----------------------------------------------------------------------------------------
%	PACKAGES AND THEMES
%----------------------------------------------------------------------------------------

\documentclass{beamer}

\mode<presentation> {%

% The Beamer class comes with a number of default slide themes
% which change the colors and layouts of slides. Below this is a list
% of all the themes, uncomment each in turn to see what they look like.

%\usetheme{default}
%\usetheme{AnnArbor}
%\usetheme{Antibes}
%\usetheme{Bergen}
%\usetheme{Berkeley}
%\usetheme{Berlin}
%\usetheme{Boadilla}
%\usetheme{CambridgeUS}
%\usetheme{Copenhagen}
%\usetheme{Darmstadt}
%\usetheme{Dresden}
%\usetheme{Frankfurt}
%\usetheme{Goettingen}
%\usetheme{Hannover}
%\usetheme{Ilmenau}
%\usetheme{JuanLesPins}
%\usetheme{Luebeck}
\usetheme{Madrid}
%\usetheme{Malmoe}
%\usetheme{Marburg}
%\usetheme{Montpellier}
%\usetheme{PaloAlto}
%\usetheme{Pittsburgh}
%\usetheme{Rochester}
%\usetheme{Singapore}
%\usetheme{Szeged}
%\usetheme{Warsaw}

% As well as themes, the Beamer class has a number of color themes
% for any slide theme. Uncomment each of these in turn to see how it
% changes the colors of your current slide theme.

%\usecolortheme{albatross}
%\usecolortheme{beaver}
%\usecolortheme{beetle}
%\usecolortheme{crane}
%\usecolortheme{dolphin}
%\usecolortheme{dove}
%\usecolortheme{fly}
%\usecolortheme{lily}
%\usecolortheme{orchid}
%\usecolortheme{rose}
%\usecolortheme{seagull}
%\usecolortheme{seahorse}
%\usecolortheme{whale}
%\usecolortheme{wolverine}

%\setbeamertemplate{footline} % To remove the footer line in all slides uncomment this line
%\setbeamertemplate{footline}[page number] % To replace the footer line in all slides with a simple slide count uncomment this line

%\setbeamertemplate{navigation symbols}{} % To remove the navigation symbols from the bottom of all slides uncomment this line
}

\usepackage{graphicx} % Allows including images
\usepackage{booktabs} % Allows the use of \toprule, \midrule and \bottomrule in tables
\usepackage{minted}
\usepackage{hyperref}

%----------------------------------------------------------------------------------------
%	TITLE PAGE
%----------------------------------------------------------------------------------------

\title[Beyond Exception Handling]{Beyond Exception Handling} % The short title appears at the bottom of every slide, the full title is only on the title page

\author{Pedro Mendes, Ricardo Pereira} % Your name
\institute[IST] % Your institution as it will appear on the bottom of every slide, may be shorthand to save space
{%
Instituto Superior Técnico \\ % Your institution for the title page
\medskip

}
\date{\today} % Date, can be changed to a custom date

\begin{document}

\begin{frame}
\titlepage% Print the title page as the first slide
\end{frame}

\begin{frame}
\frametitle{Overview} % Table of contents slide, comment this block out to remove it
\tableofcontents % Throughout your presentation, if you choose to use \section{} and \subsection{} commands, these will automatically be printed on this slide as an overview of your presentation
\end{frame}

%----------------------------------------------------------------------------------------
%	PRESENTATION SLIDES
%----------------------------------------------------------------------------------------

\section{\texttt{block} \& \texttt{return\_from}}
\begin{frame}
\frametitle{Functions \texttt{block} \& \texttt{return\_from} }

\begin{block}{}
These two functions were implemented using the try-catch mechanism present in Julia.
\end{block}

\begin{block}{\texttt{block}}
To distinguish different blocks we decided to attribute a unique id to each block, to achieve this
the \texttt{block} function uses a global counter to generate the token that will be assigned
to the block, and then used by the \texttt{return\_from} function.
If the block function catches an exception that is not of type \texttt{ReturnFrom}, it simply rethrows it
so it doesn't intervene in the rest of the program's control flow.
\end{block}

\begin{block}{\texttt{return\_from}}
The \texttt{return\_from} function then throws a known exception called \texttt{ReturnFrom} containing
the token and the return value to return in the block.
\end{block}

\end{frame}

%can add more frames to same section

%------------------------------------------------

\section{\texttt{handler\_bind}}
\begin{frame}
\frametitle{\texttt{handler\_bind}}

\begin{block}{Handler Stack}
To make the handler bind work, we used a global handler stack. This stack has a Vector of handlers, each
of them belonging to a separate \texttt{handler\_bind} call.

Using a stack of vectors instead of a flat vector of handlers allows us to nest \texttt{handler\_bind} calls
and still distinguish to which call the handler belongs to.
\end{block}

\begin{block}{\texttt{handler\_bind}}
This function then simply pushes its passed handlers to the stack and calls
the wrapped function normally.
\end{block}

\end{frame}

\subsection{\texttt{error} function}
\begin{frame}[fragile]{The \texttt{error} function when combined with \texttt{handler\_bind}}
\begin{alertblock}{Important Note}
The error function is the function that makes the whole architecture tick.
As such we will be returning to it latter in the presentation.
\end{alertblock}
\begin{block}{}
To make the \texttt{handler\_bind} work, the error function tries to find a suitable handler in the
Handler Stack, for each set of handler it tries to call all that match the type of the passed error.

Throwing in a handler will result in no more handlers being called as this counts as a
\textit{non-local transfer of control}. As such a simple implementation of it would look like this.
\end{block}
\end{frame}

\begin{frame}[fragile]{\texttt{error} example}
\begin{minted}{julia}
function error(err)
    while (hs = peek(HANDLER_STACK)) != nothing
        for h in filter(h -> err isa h.first, hs)
            h.second(err)
        end
        pop!(HANDLER_STACK)
    end
    throw(err)
end
\end{minted}
\end{frame}
%------------------------------------------------

\section{\texttt{restart\_bind}}
\begin{frame}
\frametitle{\texttt{restart\_bind}}

\begin{block}{\texttt{restart\_bind}}
\texttt{restart\_bind} is very similar to \texttt{handler\_bind}, all it does is collect
the restarts it receives and stores them in a stack (a separate one from the one used by \texttt{handler\_bind}), so
that they can be referenced later. After doing so it simply calls the wrapped function.
\end{block}

\begin{block}{Restart Stack}
We use a stack here for the same reason we used on for the handlers, this way we can pop the restarts in logical blocks
so, this also allows nesting of restarts with the same name.
\end{block}


\end{frame}

%------------------------------------------------

\section{\texttt{invoke\_restart}}

\begin{frame}
\frametitle{\texttt{invoke\_restart}}

\begin{block}{\texttt{invoke\_restart}}
Invoke restart is designed to works a possible handler of handler bind. If it is called outside this context
it's behaviour is undefined.

Given this constraint we know that handlers are only called by the error function, which means that all calls
made to this function are done indirectly by \texttt{error}.
\end{block}
\begin{block}{Finding a restart}
This function will then pop restarts from the restart stack until it finds one that matches the given
name, when it does, it calls the restart and throws another known exception called \texttt{RestartResult}
that contains the return value of the called restart. This exception is then caught by \texttt{error} which
will halt the handler lookup to return the value wrapped in the \texttt{RestartResult}
\end{block}


\end{frame}
\subsection{\texttt{error} function}
\begin{frame}{The new \texttt{error} function}
\begin{block}{Changes to the \texttt{error} function}
The \texttt{error} function had to be changed to accommodate for this feature, a try catch was added
to check if the handler threw a \texttt{RestartResult} and handle it accordingly as explained before.
\end{block}
\end{frame}
\begin{frame}[fragile]{\texttt{error} example}
\begin{minted}[fontsize=\small]{julia}
function error(err)
  while (hs = peek(HANDLER_STACK)) != nothing
    for h in filter(h -> err isa h.first, hs)
      try
        h.second(err)
      catch e # added
        # if the handler `h.second` invoked a restart
        if e isa RestartResult
          return e.result # return it
        else
          # otherwise maintain the behavior that throwing is a
          # non-local transfer of control
          rethrow(e)
    end; end; end
    pop!(HANDLER_STACK)
  end
  throw(err)
end
\end{minted}
\end{frame}

%------------------------------------------------

\section{Extensions}
\subsection{\texttt{signal}}

\begin{frame}
\frametitle{\texttt{signal}}

\begin{block}{}
The signal function simply makes use of the \texttt{Base.error} function
to abort the program no matter what is happening, throwing an \texttt{ErrorException}.
\end{block}
\end{frame}

%------------------------------------------------

\subsection{Interactive restarts}
\begin{frame}[fragile]
\frametitle{Interactive restarts}
\begin{block}{\texttt{invoke\_restart\_interactive}}
This function can be used anywhere a normal \texttt{invoke\_restart} function can be used. It then will, when called,
interrupt the program to provide an interactive prompt for the user to pick a restart or cancel to return
without invoking any restarts.
\end{block}
\begin{example}
Given this function,
\begin{minted}{julia}
reciprocal_restart(v) =
    restart_bind(:return_zero => () -> 0,
                 :return_value => identity,
                 :retry_using => (f) -> f()) do
    reciprocal(0)
end
\end{minted}
if an interactive restart is used a prompt like this will show up:
\end{example}
\end{frame}

\begin{frame}[fragile]{Interactive restarts}
\begin{example}
\begin{minted}[fontsize=\small]{text}
Choose a restart:
0: cancel :: !
1: return_zero :: Any
2: retry_using :: Any
3: return_value :: Any
restart>
\end{minted}
\end{example}
\begin{example}
If there were other restarts in previous functions they can also be called here
and are distinguished by their level on indentation.
\begin{minted}[fontsize=\small]{text}
Choose a restart:
0: cancel :: !
1: return_zero :: Any
  2: just_do_it :: Any
restart>
\end{minted}
\end{example}

\end{frame}

\begin{frame}[fragile]{Parameterized restarts}\label{f:param_restart}

\begin{block}{}
Some restarts require parameters to be passed. Because julia doesn't let you reflect on the types of
lambdas at runtime the programmer needs to supply the parameters by hand. For the example above this will look
like so
\end{block}
\begin{example}
\begin{minted}{julia}
reciprocal_restart(v) =
  restart_bind(:r_zero => () -> 0,
               :r_value => identity => (Float64,),
               :r_using => ((f) -> f()) => (Function,)) do
  reciprocal(0)
end
\end{minted}
\end{example}
\end{frame}

\begin{frame}[fragile]{Parameterized restarts}
\begin{example}
\begin{verbatim}
0: cancel :: !
1: return_zero :: Any
2: retry_using :: Function -> Any
3: return_value :: Float64 -> Any
restart> 3
Input Float64: 4.2
\end{verbatim}
\end{example}
\begin{example}
\begin{verbatim}
Choose a restart:
0: cancel :: !
1: return_zero :: Any
2: retry_using :: Function -> Any
3: return_value :: Float64 -> Any
restart> 2
Input Function: () -> 8.4 / 2
\end{verbatim}
\end{example}
\end{frame}
%------------------------------------------------
\subsection{\texttt{@handler\_case}}
\begin{frame}[fragile]
\frametitle{\texttt{@handler\_case}}

\begin{block}{Abbreviating common idioms}
This macro is a shortcut for combining \texttt{block}, \texttt{return\_from} and
\texttt{handler\_bind}.
\end{block}
\begin{example}
Instead of writing
\begin{minted}[fontsize=\small]{julia}
block() do token
  handler_bind(Except => (c) -> return_from(token, some_val)) do
    some_function()
  end
end
\end{minted}
One can simply write
\begin{minted}{julia}
@handler_case some_function() begin
    c::Except = some_val
end
\end{minted}
\end{example}

\end{frame}
\begin{frame}[fragile]{Nesting}
\begin{example}
It can also be nested in arbitrary ways without causing conflicts.
\begin{minted}{julia}
@handler_case e() begin
    c::Except1 = 1
    c::Except2 = @handler_case e() begin
            d::Except1 = c.n
            d::Except2 = d.n * 2
        end
end
\end{minted}
And even a default exception can be used much like the language's try catch expression can.

\begin{minted}{julia}
@handler_case e() begin
    c = 1
end
\end{minted}
\end{example}
\end{frame}

%------------------------------------------------
\subsection{\texttt{@fn\_types}}
\begin{frame}[fragile]{\texttt{@fn\_types}}
   \begin{block}{}
   To make this process more pleasant a new \texttt{@fn\_types} macro was implemented.
   It can be applied to lambdas whose type arguments are explicitly typed and it will generate the type tuple
   for the user.
   \end{block}
   \begin{example}
   This code is equivalent to the example in Slide~\ref{f:param_restart}.
   \begin{minted}{julia}
reciprocal_macro(v) = restart_bind(
        :return_zero => () -> 0,
        :return_value => (@fn_types (c::Float64) -> c),
        :retry_using => @fn_types (f::Function) -> f()) do
    reciprocal(0)
end
   \end{minted}
   \end{example}
\end{frame}
%------------------------------------------------

\subsection{\texttt{@restart\_case}}

\begin{frame}[fragile]
\frametitle{\texttt{@restart\_case}}
\begin{block}{}
This is a convenience macro for defining restart cases.
Instead of writing
\begin{minted}[fontsize=\small]{julia}
some_restartable(v) =
restart_bind(:return_zero => () -> 0,
             :return_value => ((c) -> c) => (Float64,),
             :retry_using => ((f) -> f()) => (Function,)) do
    some_function(v)
end
\end{minted}
One can simply write
\begin{minted}[fontsize=\small]{julia}
some_restartable(v) = @restart_case some_function(b) begin
    :return_zero => () -> 0
    :return_value => (c::Float64) -> c
    :retry_using => (f::Function) -> f()
end
\end{minted}
This macro will also make sure to include the types of the lambdas when they are typed
so that \texttt{invoke\_restart\_interactive} can then ask for the parameters for
parsing.
\end{block}
\end{frame}

%----------------------------------------------------------------------------------------
\section{Extra remarks}
\begin{frame}[fragile]{Pattern matching}
\begin{block}{\texttt{@match} macro}
During the development of the project lots of if/else-if/else chains where used theses were
very verbose but Julia does not have a builtin switch/case or pattern match mechanism.

To compensate for this a simple \texttt{@match} macro was written to alleviate this pain point.
\end{block}
\begin{example}
\begin{minted}{julia}
@match parse(Int, some_string) begin
    0   => println("Got a 0")
    1   => println("Got a 1")
    2:6 => println("Got a number between 2 and 6")
    _   => println("Got something else: \$_")
end
\end{minted}
\end{example}
\end{frame}
\begin{frame}{Pattern matching}
\begin{block}{}
This macro supports matching:
\begin{itemize}
    \item on constants;
    \item on ranges of values
    \item default cases (using the \_ character)
\end{itemize}
It was heavily inspired by the macro implemented by \href{https://github.com/kmsquire}{kmsquire} and
\href{https://github.com/kmsquire/Match.jl}{\textit{hosted on github}}.
\end{block}
\end{frame}

\end{document}
